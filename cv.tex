\documentclass{mycv}
\usepackage[UTF8, scheme = plain]{ctex}
\usepackage{setspace}
\setstretch{1.123} 
% \usepackage[onehalfspacing]{setspace}
\usepackage{enumitem}
\setenumerate[1]{itemsep=1pt,partopsep=1pt,parsep=\parskip,topsep=5pt}
\setitemize[1]{itemsep=1pt,partopsep=1pt,parsep=\parskip,topsep=5pt}
\setdescription{itemsep=1pt,partopsep=1pt,parsep=\parskip,topsep=5pt}
\name{王竟宇}
% \address{DLB 625, Department of Computer Science \\ Hong Kong Baptist University, Hong Kong}
\phone{15922267301}
\email{845064182@qq.com}
% \homepage{https://xuc.me}
\github{https://github.com/wjyzhixing}
% \phone{15922267301}
% \linkedin{xu-cheng}

\begin{document}

\maketitle%

\section{求职意向}

\subsection{Web前端开发工程师}

% \section{\rule[10pt]{18cm}{0.01em}}
\section{教育经历}
\subsection{西北农林科技大学(985/211 双一流)}[2016/09/01 – 2020/07/01 ]

\textbf{专业:电子商务(工科/计算机类)}  \hspace{1.5em}   \textbf{学历:本科}  \hspace{1.5em}   \textbf{学院:信息工程学院}   \hspace{1.5em}  \textbf{政治面貌:党员}

% \begin{itemize}
%   \item 相关课程:数据库、数据结构、算法设计、Java,计算机网络、Linux、hadoop 云计算、大数据、安卓开发、Web程序
%         设计、javaee程序设计、大学期间主修功课平均成绩80分,班级排名9/31。
%   \item 技能:对web 相关技术(html5,JavaScript,css3)熟练,了解less、sass扩展语言,熟悉ES6,7 语法,使用过 vue 框
%         架,react框架,掌握nodejs(express 框架),团队合作完成过后端项目,对计算机网络,数据结构,算
%         法有一定认知,熟练git 、svn操作,了解react native,weex等移动app开发,能进行微信小程序开发。
% \end{itemize}

\section{工作经历}

\subsection{浩鲸云计算科技股份有限公司}[2020/07/07 – 至今]

% \section{\rule[20pt]{18cm}{0.01em}}

\subsection{项目名称 :徐州健康宝(防疫码)}[2022/01/29 – 2022/03/30]
% \begin{positions}
%   \entry{Visiting Scholar}{Sep 2017~--~Dec 2017}
% \end{positions}

\begin{itemize}
  \item 职务 :前端开发工程师 (web,h5,小程序)
  \item 项目描述 : 徐州健康宝是徐州防疫码项目的c端展示名称,由后台管理平台,钉钉端h5以及微信小程序三方面组成,主要是配合政府人员亮码通信获取人员情况信息的项目。
  \item  项目小程序端人脸识别,动态表单部分模块以及部分路由跳转由本人完成,跳转苏康码相关页面嵌套为h5,由本人完成,微信小程序采用原生api,h5采用tarojs,
        后台管理采用reactjs,umijs配合antd进行菜单管理。
\end{itemize}

\subsection{项目名称 :杭州党建项目(西湖先锋)}[2021/06/22 – 2022/01/28]
% \begin{positions}
%   \entry{Visiting Scholar}{Sep 2017~--~Dec 2017}
% \end{positions}

\begin{itemize}
  \item 职务 :前端开发工程师 (web,h5,小程序)
  \item 项目描述 : 杭州党建项目是一个围绕整个杭州市所有党员需求所实现的一个系统,包含党员信息,党费上交下拨,相关企业信息,各个区
        县党员发展情况,流动党员分析,党员课程学识学习,各个服务队创建活动开展等模块。
  \item  项目学时统计,流动党员统计,志愿服务队,积分商城,今日看点,两新制图驾驶舱大屏等模块由本人编写,web端采用umijs
        配合dvajs进行项目配置,项目由reactjs框架进行完成,前端采用ant design3进行展示,接口采用umi架构connect进行前后端数据交互。h5
        模块由react配合ant design mobile进行,图表由echarts进行绘制。h5采用taro-app进行搭建,采用taro ui进行页面绘制。小程序端是采用微
        信原生的小程序进行相关web所需模块开发。
\end{itemize}

\subsection{项目名称 :重庆烟草客服子系统项目}[2021/01/03 – 2021/06/20]
% \begin{positions}
%   \entry{Ph.D.\ Candidate}{Nov 2014~--~Feb 2019}
%   \entry{Senior Research Assistant / Post-doctoral Research Fellow}{Dec 2018~--~Present}
% \end{positions}

\begin{itemize}
  \item 职务 :前端开发工程师
  \item 项目描述 : 为实现重庆烟草系统能实现在线订货功能,客服通过线上给客户电话呼出从而了解客户需求进行卷烟订货,流程上,先由烟草
        公司制定所选线路创建电话任务,客服通过呼叫完成任务,最后确定用户对烟草的需求进行完成任务的闭合,其中还涉及一些其他的呼
        入,工单等相关业务。
  \item 项目职责 : 项目电话订货,呼入弹屏,客服统计模块由本人开发,采用umijs配合dvajs进行项目配置,项目由reactjs框架进行完成,前端采
        用ant design4进行展示,接口先采取mock数据,后采用umi架构connect进行前后端数据交互。h5模块由react配合ant design mobile进行,
        图表由echarts进行绘制。
\end{itemize}

\subsection{项目名称 :杭州公安法制大屏项目}[2020/07/07 – 2020/12/31]
% \begin{positions}
%   \entry{Visiting Scholar}{Mar 2020~--~Present}
% \end{positions}

\begin{itemize}
  \item 职务 :前端开发工程师
  \item 项目描述 : 为了实现能查看杭州市公安系统内部人员相关警官执法能力,以及各个分局派出所的执法能力,所做的大屏项目。该项目从决
        策研判,模型研判,警员研判,管理研判以及质量管控模块进行整体开发分析。
  \item 项目质量管控模块由本人开发,采用umijs配合dvajs进行项目配置,项目由reactjs框架进行完成,前端采用ant design3进行展
        示,绘制图表采用bizcharts进行,接口先采取mock数据,后采用umi架构connect进行前后端数据交互。
\end{itemize}

\section{个人技能}

\begin{itemize}
      \item 对web 相关技术(html5,JavaScript,css3)熟练掌握
      \item 掌握less、sass扩展语言
      \item 熟悉ES6,7,8,9,10,11相关语法
      \item 熟悉vue框架,熟练掌握react框架,熟悉nodejs(以及对应express,koa2,eggjs 框架)
      \item 熟练掌握echarts,bizcharts等图表操作      
      \item 了解react native,weex等移动app开发,熟悉raxjs以及tarojs,熟悉原生微信小程序开发。
      \item 熟悉mysql相关语法,熟练操作navicat等数据库软件
      \item 熟悉java,对hadoop,spark大数据等有一定了解
      \item 对计算机网络,数据结构,算法有一定认知
      \item 熟悉svn操作,数量掌握git操作,有良好的分支管理经验
\end{itemize}



\section{奖励活动}

荣誉/奖项:2017获得数学建模校赛三等奖,2018年获得省级数学竞赛三等奖,英语CET4证书,2017-2018两年校图书馆借
阅排行前十,2020年度-2020届优秀毕业生-月度新星,2021年度-2020届优秀毕业生-海洋新星-非凡新兵,2022年度鲸码
奖

\section{个人评价}

热爱学习,个人的技能大部分都是课余时间自学,喜欢逛IT 博客,代码网站,研究新出现的技术,善于学习,做事情能坚
持做到底,有恒心,会将出现的问题记录下来,尽全力去找到一个解决方案,学校参加过学生会,辩论队,数学建模竞赛,
有较好的沟通能力,可以很好融入组织生活,团队意识比较好,多次和队伍一起参与各种比赛,自己也有结题的大学生校级
科技创新项目;在公司能不遗余力投入项目开发,在各个项目能收到较强正向反馈,获得较多表彰。

% \subsection{Homebrew}[Hong Kong]
% \begin{positions}
%   \entry{Core Maintainer}{Feb 2015~--~Feb 2017}
% \end{positions}

% \begin{itemize}
%   \item \url{https://brew.sh}
%   \item Acted as one of the core maintainers for the open source project Homebrew, which is the most popular package manager on macOS\@.
%   \item Implemented several major features and improvements including better tap system, core/formulae split, sandbox system, portable Ruby, and many bug fixes.
% \end{itemize}

% \section{Education}

% \subsection{Hong Kong Baptist University}[Hong Kong]
% \vspace{-\parskip}%
% \begin{itemize}[label={}]
%   \item Ph.D.\ in Computer Science \printdate{Nov 2014~--~Feb 2019}
%   \item Dissertation: \href{https://repository.hkbu.edu.hk/etd_oa/620}{Authenticated Query Processing in the Cloud}
%   \item Advisor: \href{https://www.comp.hkbu.edu.hk/~xujl}{Prof.~Jianliang Xu}
% \end{itemize}

% \subsection{Huazhong University of Science and Technology}[Wuhan, China]
% \vspace{-\parskip}%
% \begin{itemize}[label={}]
%   \item Bachelor of Engineering in Electronics \& Information Engineering \printdate{Sep 2009~--~Jun 2014}
% \end{itemize}

% \section{Skills}

% \begin{description}
%   \item[Programming] C/C++, Rust, Java, Python, Ruby, Matlab, \LaTeX, Bash, Javascript
%   \item[Tools] Vim, Tmux, Git, macOS, Linux
%   \item[Languages] English, Mandarin
% \end{description}

% \section{Publications}%

% \publications{publications.bib}

% \section{Talks}

% \begin{enumerate}
%   \item Blockchain Privacy Preserving Techniques, \emph{The 36th CCF National Database Conference}, Jinan, China, Oct.~2019.
%   \item Towards Searchable and Verifiable Blockchain, \emph{1st Workshop on Blockchain and Data Management at 35th IEEE International Conference on Data Engineering}, Macau, Apr.~2019.
%   \item When Query Authentication Meets Fine-Grained Access Control: A Zero-Knowledge Approach, \emph{2018 ACM SIGMOD International Conference on Management of Data}, Houston, USA, Jun.~2018.
% \end{enumerate}

% \section{Awards}

% \begin{itemize}
%   \item SIGMOD Travel Award, ACM \printdate{2018}
%   \item Department RPg Performance Award, Hong Kong Baptist University \printdate{2018,~2019}
%   \item Postgraduate Research Symposium Best Research Performance Award \& Best Poster Award, Hong Kong Baptist University \printdate{2018}
%   \item Yakun Scholarship Scheme for Mainland Postgraduate Students, Hong Kong Baptist University \printdate{2018}
%   \item Excellent Teaching Assistant Performance Award, Hong Kong Baptist University \printdate{2017}
%   \item Teaching Assistant Performance Award, Hong Kong Baptist University \printdate{2015,~2016}
% \end{itemize}

\end{document}
